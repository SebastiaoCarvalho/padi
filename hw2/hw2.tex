\documentclass{article}

\usepackage{amsmath}
\usepackage{amssymb}
\usepackage{tikz}
\usetikzlibrary{shapes,arrows,positioning}
\renewcommand{\thesubsection}{\thesection \space \alph{subsection})}

\title{Planing, Learning and Intelligent Decision Making - Homework 2}
\author{99326 - Sebastião Carvalho, 99331 - Tiago Antunes}
\date{\today}

\begin{document}

\maketitle

\tableofcontents

\section{Question 1}

\subsection{}

Considering $\mathcal{X}$ as our state space, $\mathcal{X} = \{1P, 2P, 2NP, 3P, 3NP, 4P, 4NP\}$, 
where each state represents the corresponding cell and if they have the passanger or not (P or NP). 
Since when we are on the first cell we always have the passanger, state we dont consider $1NP$.

Considering $\mathcal{A}$ as our action space, $\mathcal{A} = \{U, D, L, R\}$, 
where each action represents the movement of the agent, up, down, left and right, respectively.

\subsection{}

The transition matrix for the action down, $P_D$, is given by
$\begin{bmatrix}
    0.2 & 0 & 0 & 0.8 & 0 & 0 & 0 \\
    0 & 0.2 & 0 & 0 & 0 & 0.8 & 0 \\
    0 & 0 & 0.2 & 0 & 0 & 0 & 0.8 \\
    0.4 & 0 & 0 & 0.2 & 0 & 0.4 & 0 \\
    0.4 & 0 & 0 & 0 & 0.2 & 0 & 0.4 \\
    0 & 0.4 & 0 & 0.4 & 0 & 0.2 & 0 \\
    0 & 0 & 0.4 & 0 & 0.4 & 0 & 0.2 \\
\end{bmatrix}$

\medskip

, where each row represents the state we are in and each column represents the state we are going to.

\bigskip

Since we want a simple cost function, we'll consider that the movement to each state costs 1,
and the movement to the goal state ($2P$) costs 0. So, the cost matrix for all actions is given by
$\begin{bmatrix}
    1 & 1 & 1 & 1 \\
    1 & 1 & 1 & 1 \\
    0 & 1 & 0 & 0 \\
    1 & 1 & 1 & 1 \\
    1 & 1 & 1 & 1 \\
    0 & 1 & 1 & 1 \\
    1 & 1 & 1 & 1 \\
\end{bmatrix}$

\medskip

, where each row represents the state we are in and each column represents the action we are taking.

\medskip

Basically, $c(x, a) = \left\{\begin{array}{ll}
    0 & \text{if } (x, a) = (4P, U), (2P, U), (2P, R), (2P, L) \\
    1 & \text{else}
\end{array}
\right.$

\subsection{}

\end{document}